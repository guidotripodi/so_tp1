
\subsubsection{Introduccion}

\subsubsection{Ejercicios}
\begin{itemize}
 \item 
\textbf{Ejercicio 3}  Completar la implementacion del scheduler Round-Robin implementando los
metodos de la clase SchedRR en los archivos sched rr.cpp y sched rr.h. La implementacion
recibe como primer parametro la cantidad de nucleos y a continuacion los valores de sus
respectivos quantums. Debe utilizar una unica cola global, permitiendo ası la migracion de
procesos entre nucleos.
\item \textbf{Ejercicio 4} Diseñar uno o mas lotes de tareas para ejecutar con el algoritmo del ejercicio
anterior. Graficar las simulaciones y comentarlas, justificando brevemente por que el comportamiento 
observado es efectivamente el esperable de un algoritmo Round-Robin.
\item \textbf{Ejercicio 5} A partir del artıculo
Waldspurger, C.A. and Weihl, W.E., Lottery scheduling: Flexible proportional-share re-
source management. Proceedings of the 1st USENIX conference on Operating Systems
Design and Implementation – 1994.
diseñar e implementar un scheduler basado en el esquema de loterıa. El constructor de la
clase SchedLottery debe recibir dos parametros: el quantum y la semilla de la secuencia
pseudoaleatoria (en ese orden). Interesa implementar al menos la idea basica del algoritmo
y la optimizacion de tickets compensatorios (compensation tickets). Otras optimizaciones y
refinamientos que propone el artıculo seran opcionales siempre que, en cada caso, se explique
brevemente por que la optimizacion no se considero relevante a los efectos de este trabajo.

\end{itemize}


\subsubsection{Resultados y Conclusiones}
